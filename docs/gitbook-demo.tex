% Options for packages loaded elsewhere
\PassOptionsToPackage{unicode}{hyperref}
\PassOptionsToPackage{hyphens}{url}
%
\documentclass[
]{book}
\usepackage{amsmath,amssymb}
\usepackage{iftex}
\ifPDFTeX
  \usepackage[T1]{fontenc}
  \usepackage[utf8]{inputenc}
  \usepackage{textcomp} % provide euro and other symbols
\else % if luatex or xetex
  \usepackage{unicode-math} % this also loads fontspec
  \defaultfontfeatures{Scale=MatchLowercase}
  \defaultfontfeatures[\rmfamily]{Ligatures=TeX,Scale=1}
\fi
\usepackage{lmodern}
\ifPDFTeX\else
  % xetex/luatex font selection
\fi
% Use upquote if available, for straight quotes in verbatim environments
\IfFileExists{upquote.sty}{\usepackage{upquote}}{}
\IfFileExists{microtype.sty}{% use microtype if available
  \usepackage[]{microtype}
  \UseMicrotypeSet[protrusion]{basicmath} % disable protrusion for tt fonts
}{}
\makeatletter
\@ifundefined{KOMAClassName}{% if non-KOMA class
  \IfFileExists{parskip.sty}{%
    \usepackage{parskip}
  }{% else
    \setlength{\parindent}{0pt}
    \setlength{\parskip}{6pt plus 2pt minus 1pt}}
}{% if KOMA class
  \KOMAoptions{parskip=half}}
\makeatother
\usepackage{xcolor}
\usepackage{longtable,booktabs,array}
\usepackage{calc} % for calculating minipage widths
% Correct order of tables after \paragraph or \subparagraph
\usepackage{etoolbox}
\makeatletter
\patchcmd\longtable{\par}{\if@noskipsec\mbox{}\fi\par}{}{}
\makeatother
% Allow footnotes in longtable head/foot
\IfFileExists{footnotehyper.sty}{\usepackage{footnotehyper}}{\usepackage{footnote}}
\makesavenoteenv{longtable}
\usepackage{graphicx}
\makeatletter
\def\maxwidth{\ifdim\Gin@nat@width>\linewidth\linewidth\else\Gin@nat@width\fi}
\def\maxheight{\ifdim\Gin@nat@height>\textheight\textheight\else\Gin@nat@height\fi}
\makeatother
% Scale images if necessary, so that they will not overflow the page
% margins by default, and it is still possible to overwrite the defaults
% using explicit options in \includegraphics[width, height, ...]{}
\setkeys{Gin}{width=\maxwidth,height=\maxheight,keepaspectratio}
% Set default figure placement to htbp
\makeatletter
\def\fps@figure{htbp}
\makeatother
\setlength{\emergencystretch}{3em} % prevent overfull lines
\providecommand{\tightlist}{%
  \setlength{\itemsep}{0pt}\setlength{\parskip}{0pt}}
\setcounter{secnumdepth}{5}
\usepackage{booktabs}
\usepackage{amsthm}
\makeatletter
\def\thm@space@setup{%
  \thm@preskip=8pt plus 2pt minus 4pt
  \thm@postskip=\thm@preskip
}
\makeatother
\ifLuaTeX
  \usepackage{selnolig}  % disable illegal ligatures
\fi
\usepackage[]{natbib}
\bibliographystyle{apalike}
\usepackage{bookmark}
\IfFileExists{xurl.sty}{\usepackage{xurl}}{} % add URL line breaks if available
\urlstyle{same}
\hypersetup{
  pdftitle={Linear Models},
  pdfauthor={Mohammad Arashi},
  hidelinks,
  pdfcreator={LaTeX via pandoc}}

\title{Linear Models}
\author{\href{http://prof.um.ac.ir/arashi/}{Mohammad Arashi}}
\date{2025-02-28}

\begin{document}
\maketitle

{
\setcounter{tocdepth}{1}
\tableofcontents
}
\chapter{Generalized Inverses}\label{generalized-inverses}

\textbf{Definition}: A generalized inverse of the matrix

(In algebra, a system of equations (either linear or nonlinear) is
called consistent if there is at least one set of values for the
unknowns that satisfies each equation in the system---that is, when
substituted into each of the equations, they make each equation hold
true as an identity.)

\section{}\label{section}

\chapter{Chapter 2}\label{chapter2}

\chapter{Chapter 3}\label{chapter-3}

\chapter{Chapter 4}\label{ch4}

\chapter{Chapter 5}\label{chapter-5}

\chapter{Chapter 6}\label{chapter-6}

\chapter{Chapter 7}\label{chapter-7}

\chapter{License}\label{license}

\begin{itemize}
\tightlist
\item
  \href{https://creativecommons.org/share-your-work/public-domain/cc0/}{CC0 (``No Rights Reserved'')}, everybody can do what they want with your work.
\item
  \href{https://creativecommons.org/licenses/by/4.0/}{CC-BY 4.0 (``Attribution'')}, everybody can do what they want with your work, but they must credit you. Note that this license may not be suitable for software or source code!
\end{itemize}

For compatibility between CC and GNU licenses, see \href{https://creativecommons.org/faq/\#Can_I_apply_a_Creative_Commons_license_to_software.3F}{this FAQ}.

\end{document}
